\documentclass[10pt,slidestop,mathserif]{beamer}
%[slidestop] puts frame titles & contents on the top left corner (default = [slidescentered])
%[red] changes navigation bars and title to reddish color
%[mathserif] use serif fonts for representing formulas instead of sans serif (default = mathsans)
%[notes] adds notes to PDF screen
%[compress] the navigationbar in one line
%[notesonly] make only notes
%\mode<presentation>
\usepackage{xeCJK}
\usepackage{hyperref}
%\usepackage{amsmath} %for math AMS fonts
%\usepackage{graphicx} %to include figures
%\usepackage{subfigure} %to have figures in figures
%\usepackage{multimedia} %to include movies


%Fonts
\setCJKmainfont[BoldFont={Adobe Heiti Std}, ItalicFont={Adobe Kaiti Std}]{Adobe Song Std}
\setCJKsansfont{Adobe Kaiti Std}
\setCJKmonofont{Adobe Fangsong Std}
\setmainfont[Mapping=tex-text]{Liberation Serif}
\setsansfont{Liberation Sans}
\setmonofont{Bitstream Vera Sans Mono}


%NewCommands
\newcommand\abs[1]{\left\lvert #1 \right\rvert}
\newcommand\floor[1]{\left\lfloor #1 \right\rfloor}
\newcommand\ceil[1]{\left\lceil #1 \right\rceil}
\newcommand\yin[1]{\textit{#1}}
\newcommand\yang[1]{\textbf{#1}}


%Theme
%View on website: http://www.hartwork.org/beamer-theme-matrix/
\usetheme{Warsaw}
\usecolortheme{default}
%\useoutertheme[subsection=false]{smoothbars}
%\useinnertheme{rectangles}


%Style
\setbeamertemplate{footline}{}
%\beamertemplatetransparentcoveredhigh %makes all covered text highly transparent
%\beamertemplatetransparentcovereddynamicmedium %makes all covered text quite transparent, but is a dynamic way. The range of dynamics is smaller






%Animation
%\beamerdefaultoverlayspecification{<+->}
%Read Amber M. Smith




%PDF
%\hypersetup{pdfpagemode=FullScreen}



\begin{document}

%%Fix the bug of guide button; Beamer 3.17+ fixed!
\makeatletter
\def\beamer@linkspace#1{%
	\begin{pgfpicture}{0pt}{-1.5pt}{#1}{5.5pt}
		\pgfsetfillopacity{0}
		\pgftext[x=0pt,y=-1.5pt]{.}
		\pgftext[x=#1,y=5.5pt]{.}
	\end{pgfpicture}
}
\makeatother


\title{Example}
\subtitle{Beamer}
\author{{\scriptsize\bfseries Yinyanghu} \\ {\scriptsize\bfseries 3.141...}}
\institute{NJU}
\subject{Computer Science}
\date{\today}

%\title[Crisis] % (optional, only for long titles)
%{The Economics of Financial Crisis}
%\subtitle{Evidence from India}
%\author[Author, Anders] % (optional, for multiple authors)
%{F.~Author\inst{1} \and S.~Anders\inst{2}}
%\institute[Universitäten Hier und Dort] % (optional)
%{
%	\inst{1}%
%	Institut für Informatik\\
%	Universität Hier
%	\and
%	\inst{2}%
%	Institut für theoretische Philosophie\\
%	Universität Dort
%}
%\date[KPT 2004] % (optional)
%{Konferenz über Präsentationstechniken, 2004}
%\subject{Informatik}

%[plain] for plane frame style
%[containsverbatim] for using verbatim environment and \verb command
%[allowframebreaks] for automatic split of frames if the contents do not fit in a single slide
%[shrink] for shrinking the contents to fit in a single slide
%[squeeze] for squeezing vertical space
\begin{frame}[plain]
	\titlepage
\end{frame}

\begin{frame}
	\tableofcontents
\end{frame}

\section{A}
\subsection{A1}
\begin{frame}[shrink]
	\frametitle{A1}
	\framesubtitle{test}

	Erd\H os
	\begin{block}{P}
		\begin{itemize}
			\item A %\pause
			\item B %\pause
			\item C %\pause
		\end{itemize}
	\end{block}

	\footnote{On a fast machine.}
\end{frame}

\section{B}

\subsection{B1}
\begin{frame}
	\frametitle{B1}
	\begin{columns}[t]
		\column{0.33\textwidth}
		. . . contents A . . .
		\column{0.33\textwidth}
		. . . contents B . . .
		\column{0.33\textwidth}
		. . . contents C . . .
	\end{columns}
\end{frame}

\subsection{B2}
\begin{frame}
	\frametitle{B2}
	\label{th1}
	\begin{theorem}
		\begin{equation}
			e^{i\pi} = -1
		\end{equation}
	\end{theorem}

	\begin{proof}
		\begin{equation}
			A = B + C
		\end{equation}
	\end{proof}

	\begin{example}
		Write your fantastic \\
	\end{example}

	\begin{definition}
		Write your fantastic \\
	\end{definition}

	%corollary, alertblock

\end{frame}

\section{C}
\begin{frame}[containsverbatim]
	\frametitle{C1}
	\begin{verbatim}

         F()
A -----------------> B
         G()
	\end{verbatim}
\end{frame}

\begin{frame}
	\frametitle{C2}
	\hyperlink{th1}{\beamergotobutton{Jump to Theorem \#1}}
	\hypertarget{th1}{help}
\end{frame}

\begin{frame}
	\frametitle{C3}
	\begin{itemize}
		\item <+-| alert@+> Every thing
		\item <+-| alert@+> that has
		\item <+-| alert@+> beginning
		\item <+-| alert@+> has end.
	\end{itemize}

	\begin{itemize}
		\item<2-> \alt<2>{\color{blue}Everything}{\color{gray} Everything}
		\item<2-> \alt<3>{\color{blue}that has}{\color{gray} that has}
		\item<2-> \alt<4>{\color{blue}beginning}{\color{gray} beginning}
		\item<2-> \alt<5>{\color{blue}has end.}{\color{gray} has end.}
	\end{itemize}
				
			
		
	

\end{frame}

%Overlay
%\pause
%

\end{document}
