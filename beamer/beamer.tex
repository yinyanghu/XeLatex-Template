\documentclass[10pt,slidestop,mathserif]{beamer}
%[slidestop] puts frame titles & contents on the top left corner (default = [slidescentered])
%[red] changes navigation bars and title to reddish color
%[mathserif] use serif fonts for representing formulas instead of sans serif (default = mathsans)
%[notes] adds notes to PDF screen
%[compress] the navigationbar in one line
%[notesonly] make only notes
%\mode<presentation>
\usepackage{xeCJK}
\usepackage{hyperref}
%\usepackage{amsmath} %for math AMS fonts
%\usepackage{graphicx} %to include figures
%\usepackage{subfigure} %to have figures in figures
%\usepackage{multimedia} %to include movies


%Configuration

\lstset{aboveskip=0.2em,
  breaklines=true,
  commentstyle={\color{red}},
  escapeinside={``},
  floatplacement=tbp,
  frame=topline,
  keywordstyle={\color{blue}\ttfamily},
  %language=Java,
  numbers=left,
  numberstyle={\small},
  tabsize=2,
  xleftmargin=2em,
  xrightmargin=0em}

\setcounter{secnumdepth}{2}
\setcounter{tocdepth}{2}

%\onehalfspacing
%\providecommand{\tabularnewline}{\\}


\makeatletter
\let\@afterindentfalse
\@afterindenttrue
\@afterindenttrue
\makeatother


%PageStyle
\geometry{verbose,tmargin=25mm,bmargin=25mm,lmargin=14.5mm,rmargin=14.5mm}
\pagestyle{fancy}
\fancyhead{}%clear all settings
\fancyfoot{}%clear all settings
\fancyhead[RO,LE]{\thepage}
\renewcommand{\headrulewidth}{0.7pt}
\renewcommand{\baselinestretch}{1.2}
\XeTeXlinebreaklocale "zh"
\XeTeXlinebreakskip = 0pt plus 1pt minus 0.1pt
\renewcommand{\arraystretch}{1.5}
\setlength{\parindent}{2em}


%Sign
\renewcommand{\labelenumi}{\alph{enumi})}


%Fonts
\setmainfont{Times New Roman}
\setsansfont{Candara}
\setCJKmainfont[BoldFont=Adobe Heiti Std,ItalicFont=Adobe Kaiti Std]{Adobe Song Std}



%Name
\renewcommand{\contentsname}{目录}
\renewcommand{\listfigurename}{插图目录}
\renewcommand{\listtablename}{表格目录}
\renewcommand{\refname}{参考文献}
\renewcommand{\abstractname}{摘要}
\renewcommand{\indexname}{索引}
\renewcommand{\tablename}{表格}
\renewcommand{\figurename}{图}


%NewCommand
\newcommand{\ChineseTitle}[1]{~\par\vspace{6pt}{\fontsize{20pt}{\baselineskip}\bf\selectfont#1}\par\vspace{18pt}}
\newcommand{\ChineseAuthor}[1]{{\fontsize{14pt}{\baselineskip}\selectfont#1}\par\vspace{16pt}}
\newcommand{\ChineseInstitude}[1]{{\fontsize{10pt}{\baselineskip}\selectfont#1}\par\vspace{16pt}}
\newcommand{\EnglishTitle}[1]{{\fontsize{16pt}{\baselineskip}\bf\selectfont#1}\par\vspace{12pt}}
\newcommand{\EnglishAuthor}[1]{{\fontsize{10pt}{\baselineskip}\selectfont#1}\par\vspace{12pt}}
\newcommand{\EnglishInstitude}[1]{{\fontsize{9pt}{\baselineskip}\selectfont#1}\par\vspace{28pt}}
\newcommand{\EnglishAbstract}[1]{{\fontsize{10pt}{\baselineskip}\selectfont{\bf Abstract:~}#1}\par\vspace{12pt}}
\newcommand{\EnglishKeywords}[1]{{\fontsize{10pt}{\baselineskip}\selectfont{\bf Keywords:~}#1}\par\vspace{12pt}}
\newcommand{\ChineseAbstract}[1]{{\textbf{摘~~要:~~}}{\emph{#1}}\par\vspace{12pt}}
\newcommand{\ChineseKeywords}[1]{\textbf{关键词:~~}{#1}\par\vspace{12pt}}
\newcommand{\yin}[1]{\textif{#1}}
\newcommand{\yang}[1]{\textbf{#1}}



\begin{document}

%\input{fixbug.tex}

\title{Example}
\subtitle{Beamer}
\author{{\scriptsize\bfseries Yinyanghu} \\ {\scriptsize\bfseries 3.141...}}
\institute{NJU}
\subject{Computer Science}
\date{\today}

%\title[Crisis] % (optional, only for long titles)
%{The Economics of Financial Crisis}
%\subtitle{Evidence from India}
%\author[Author, Anders] % (optional, for multiple authors)
%{F.~Author\inst{1} \and S.~Anders\inst{2}}
%\institute[Universitäten Hier und Dort] % (optional)
%{
%	\inst{1}%
%	Institut für Informatik\\
%	Universität Hier
%	\and
%	\inst{2}%
%	Institut für theoretische Philosophie\\
%	Universität Dort
%}
%\date[KPT 2004] % (optional)
%{Konferenz über Präsentationstechniken, 2004}
%\subject{Informatik}

%[plain] for plane frame style
%[containsverbatim] for using verbatim environment and \verb command
%[allowframebreaks] for automatic split of frames if the contents do not fit in a single slide
%[shrink] for shrinking the contents to fit in a single slide
%[squeeze] for squeezing vertical space
\begin{frame}[plain]
	\titlepage
\end{frame}

\begin{frame}
	\tableofcontents
\end{frame}

\section{A}
\subsection{A1}
\begin{frame}[shrink]
	\frametitle{A1}
	\framesubtitle{test}

	Erd\H os
	\begin{block}{P}
		\begin{itemize}
			\item A %\pause
			\item B %\pause
			\item C %\pause
		\end{itemize}
	\end{block}

	\footnote{On a fast machine.}
\end{frame}

\section{B}

\subsection{B1}
\begin{frame}
	\frametitle{B1}
	\begin{columns}[t]
		\column{0.33\textwidth}
		. . . contents A . . .
		\column{0.33\textwidth}
		. . . contents B . . .
		\column{0.33\textwidth}
		. . . contents C . . .
	\end{columns}
\end{frame}

\subsection{B2}
\begin{frame}
	\frametitle{B2}
	\label{th1}
	\begin{theorem}
		\begin{equation}
			e^{i\pi} = -1
		\end{equation}
	\end{theorem}

	\begin{proof}
		\begin{equation}
			A = B + C
		\end{equation}
	\end{proof}

	\begin{example}
		Write your fantastic \\
	\end{example}

	\begin{definition}
		Write your fantastic \\
	\end{definition}

	%corollary, alertblock

\end{frame}

\section{C}
\begin{frame}[containsverbatim]
	\frametitle{C1}
	\begin{verbatim}

         F()
A -----------------> B
         G()
	\end{verbatim}
\end{frame}

\begin{frame}
	\frametitle{C2}
	\hyperlink{th1}{\beamergotobutton{Jump to Theorem \#1}}
	\hypertarget{th1}{help}
\end{frame}

\begin{frame}
	\frametitle{C3}
	\begin{itemize}
		\item <+-| alert@+> Every thing
		\item <+-| alert@+> that has
		\item <+-| alert@+> beginning
		\item <+-| alert@+> has end.
	\end{itemize}

	\begin{itemize}
		\item<2-> \alt<2>{\color{blue}Everything}{\color{gray} Everything}
		\item<2-> \alt<3>{\color{blue}that has}{\color{gray} that has}
		\item<2-> \alt<4>{\color{blue}beginning}{\color{gray} beginning}
		\item<2-> \alt<5>{\color{blue}has end.}{\color{gray} has end.}
	\end{itemize}
				
			
		
	

\end{frame}

%Overlay
%\pause
%

\end{document}
